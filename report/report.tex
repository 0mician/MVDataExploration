\documentclass[11pt, a4paper,titlepage]{article}
\usepackage[utf8]{inputenc}
\usepackage{setspace}
\usepackage{longtable}
\usepackage{textpos}
\usepackage{acronym}
\usepackage[nonumberlist,acronym,toc]{glossaries}
\usepackage{latexsym}
\usepackage{textcomp}
\usepackage[bottom]{footmisc}
\usepackage[left=2.85cm, right=2.85cm, top=2.85cm, bottom=2.85cm]{geometry}
\usepackage[english]{babel}
\usepackage{graphicx}
\graphicspath{ {figures/} }
\usepackage{array}
\usepackage{float}
\usepackage{pdfpages}
\usepackage{titlesec}
\usepackage{siunitx}
\usepackage{enumitem}
\usepackage{adjustbox}
\usepackage[small]{caption}
\usepackage{subcaption}
\usepackage{lscape}
\usepackage[toc,page]{appendix}
\usepackage[pdftex, pdfborderstyle={/S/U/W 0} % this disables the boxes around links
            ]{hyperref}

% Equation environment
\newenvironment{conditions}
  {\par\vspace{\abovedisplayskip}\noindent\begin{tabular}{>{$}l<{$} @{${}={}$} l}}
  {\end{tabular}\par\vspace{\belowdisplayskip}}

\begin{document}

\begin{titlepage}
  \begin{center}
    
    \includegraphics[scale=1.5]{figures/kuleuven_logo.pdf}~\\[4.5cm]
    \textsc{\Large Master of bioinformatics}\\[0.5cm]

    % Title
    \rule{\linewidth}{0.3mm}\\[0.4cm]
    {\huge \bfseries Applied multivariate statistical analysis} \\[0.4cm]
    {\large Multivariate dataset exploration: genome assembly} \\[0.4cm]
    \rule{\linewidth}{0.3mm}\\[0.4cm]
    {\large Summer 2016} \\[1.0cm]
    
    % Author and supervisor
    \begin{minipage}{0.4\textwidth}
      \begin{flushleft} \large
        \emph{Author:}\\
	Cedric \textsc{Lood}
      \end{flushleft}
    \end{minipage}
%     %\hfill
    \begin{minipage}{0.4\textwidth}
      \vspace{25pt}
      \begin{flushright} \large
        \emph{Teacher:} \\
        Prof. Eddie \textsc{Schrevens}\\
        \hfill \newline 
      \end{flushright}
    \end{minipage}
    
    \vfill

    \includegraphics[scale=0.15]{figures/KUL.jpg}~\\[0cm]

    % Bottom of the page
    {\large \today}
    
  \end{center}
\end{titlepage}


\section{Context of this project}

The analysis presented in this report was produced for the class of
\emph{Applied multivariate statistics} taught at KU Leuven (Winter
2015). The requirement for the class included the exploration of a
multivariate dataset of our choice in order to discover its
structure. The implementation was done using the R programming
environment (v3.3.0), and the dataset along with the code can be found
online in my github
account\footnote{https://github.com/Milt0n/MVDataExploration}.

\section{Genome assembly: quality control metrics}

The dataset consists of quality control metrics for \emph{de novo}
genome assembly \cite{baker2012novo}. It originates from an analysis
performed in the early stages of my master thesis in which I
investigated the genomics of a set of 47 nosocomial isolates of the
bacterial species \emph{Pseudomonas aeruginosa}.

The goal of genome assembly is to reconstruct the genome of an
organism using the reads issued by a sequencer, which in my case was
an \emph{Illumina} machine. For each of the 47 strains, multiple
genome assemblies were performed using different software and
approaches. \emph{De novo} genome assemblies can be thought of as
hypothesis as to what the genome of the organism looks like, and some
quality metrics can be used to evaluate their quality.

\section{Description of the dataset}

The dataset consists of 3102 observations, as each of the 47 strains
went through 67 different assemblies each, and 36 variables were
surveyed. A few missing values exist in the dataset, but their amount
is very limited. An exhaustive description of the variables is
available here \cite{gurevich2013quast}, here is a summary of selected
variables:

\begin{table}[h]
  \centering
  \begin{tabular}{|l|l|l|}
    \hline
    \# & name            & description                                                  \\ \hline
    1  & Strain ID       & Label with the ID of the strain (from 9108 to 9154)          \\
    2  & Coverage        & Genome coverage estimation based on sequencing results       \\
    3  & Assembly        & Label for the 67 assembly pipelines                          \\
    4  & Hybrid          & Boolean value indicating the approach for the assembly       \\ 
    5  & NContig         & Total number of contigs for the assembly                     \\
    6  & LargestContig   & Length (in base pairs) of the longuest contig                \\
    7  & TotalLength     & Total length of the assembled genome                         \\ 
    8  & ReferenceLength & Length of reference genome used for QC evaluation            \\ 
    9  & GC              & GC content of the assembled genome (\%)                      \\ 
    10 & ReferenceGC     & GC content of reference genome used for QC evaluation        \\
    11 & N50             & Minimum length of contig comprising 50\% of assembled genome \\ 
    12 & NG50            & Corrected N50 using the length of reference genome           \\ 
    13 & N75             & Minimum length of contig comprising 75\% of assembled genome \\ 
    15 & L50             & Number of contigs of length greater than N50                 \\ 
    19 & Nmisassemblies  & Number of misassemblies events                               \\ 
    24 & GenomeFraction  & Fraction of reference genome covered by assembly             \\ 
    29 & NA50            & Corrected N50 taking into account misassemblies              \\ 
    30 & NGA50           & Corrected NG50 taking into account misassemblies             \\ 
    36 & LGA75           & Corrected LG75 taking into account misassemblies             \\ \hline
  \end{tabular}
\end{table}

\bibliographystyle{ieeetr} 
\bibliography{bib-db}

\end{document}